\documentclass{article}
\usepackage{graphicx}
\begin{document}

\author{Andrew Bartnof}
\title{Difference-in-Difference Study for GGRF}
\date{\today}
\maketitle

\section{Introduction}

This paper documents my difference-in-difference (\emph{DiD}) analyses for the GGRF insurance project, in which I attempt to quantify the impact of a green energy project on a neighborhood's home appreciation rates.
I present a mixed-effects model in the main part of this paper.
In the appendix, I present another model, which uses a more conventional two-way fixed-effect DiD schema.

\section{Brief Background}

If green energy projects (eg solar farms and wind turbines) are built near residential areas, these projects can impact local home values, lowering the rate at which they appreciate vis-\`a-vis homes in other areas.
This gives home-owners an incentive to resist green projects in their neighborhoods, even if they support green projects in general.

We want to quantify this potential relative change in the rate in which homes appreciate near green energy projects, so that we can help to develop a new kind of financial project; a parametric risk insurance payout for home-owners near green energy projects.
The idea is that, for a limited amount of time, while the green energy projects artificially depress local homes' values, any home-owner that sells their house can be compensated for this drop in home-value.


\section{Data}

Naturally, there are a few dimensions we'd like to better understand-- do all green energy projects lower the rates at which homes appreciate? how far need a house be from a project before its value is impacted? and how long does this change in home value endure?
This is a rather 'cheap-and-cheerful' analysis using readily-available datasets, so I'll attempt to answer these, while making sensible assumptions where necessary.

In this paper, I'll  refer to homes that are near green energy projects as \textbf{manipulation} homes, and homes that are not near green energy projects as \textbf{control} homes.

I use three key datasets:

\subsection{Zillow Home Prices}
This dataset lists the home prices for 26,344 zip-codes, between the years 2000 and 2024.
This was collected before I jumped on the GGRF project so I can't give much more insight into it.

\subsection{Large-Scale Solar Photovoltaic Sites}
The solar dataset\footnote{Fujita, K.S., Ancona, Z.H., Kramer, L.A. et al. Georectified polygon database of ground-mounted large-scale solar photovoltaic sites in the United States. Sci Data 10, 760 (2023). https://doi.org/10.1038/s41597-023-02644-8} lists the locations and years of operation of thousands of large-scale solar photovoltaic sites.
Note that Ben Hoen is an author of both this dataset and the turbine dataset.

\subsection{Wind Turbines}
This dataset\footnote{
Rand, J.T., Kramer, L.A., Garrity, C.P. et al. A continuously updated, geospatially rectified database of utility-scale wind turbines in the United States. Sci Data 7, 15 (2020). https://doi.org/10.1038/s41597-020-0353-6} lists the locations and years of operation of thousands of wind turbines.

\subsection{Data Notes}
Each observation in the solar and turbine datasets is represented as a polygon in a shapefile.
The Zillow file commits us, however, to using zip-codes as our finest geographic granularity.
Consequently, I assign each solar and turbine project to a single zip-code. 
Any homes that share a zip-code with a green project are considered manipulations.
(We'd prefer not to say that a home either \emph{is}, or \emph{isn't}, near a solar project as if it's a binary metric, but it'll do.)

I also round all dates to the year.
If there are multiple home values noted in a given zip-code within this year, I take the median home price.

\section{Mixed-Effects Model}

The formula for our mixed-effects model looks like this:
\begin{verbatim}
rate of home appreciation ~ 
\end{verbatim}
\begin{verbatim}
year + solar_manipulation + turbine_manipulation + (1|zip_code)
\end{verbatim}

\textbf{Dependent variable}
\begin{itemize}
\item The rate of home appreciation is calculated as: \emph{(current home price - last year's home price) / (last year's home price).}
\end{itemize}

\textbf{Fixed effects}
\begin{itemize}
\item Year: the calendar year, represented as an unordered factor. This allows us to represent general market fluctuations (eg the Great Recession) easily.
\item Solar manipulation: if a zip code ever had a solar plant built there, we encode the zip-code each year as one of the following:
	\begin{itemize}
	\item Construction: if only one solar plant is built in this zip-code, then 'construction' is synonymous with 'year of operation'. If multiple solar plants are built in this zip-code, this represents the duration between the first plant's year of operation and the last plant's year of operation. 'Construction' indicates that we have faith that during this time period, workers and trucks were around.
	\item -5:-1: Years prior to year of first plant's operation
	\item 1:5: Likewise, years after last plant's operation
	\item Censored: if we are more than five years before construction, or less than five years after, construction in a zip-code
	\item Control: a zip-code that never gets a solar project
	\end{itemize}
\item Turbine manipulation: encoded exactly like solar manipulation
\end{itemize}
\textbf{Random effects}
\begin{itemize}
\item Zip-code: here, I model all zip-codes as random intercepts. This is common practice in models that have repeated measures per subject; this compensates for the mean difference in home appreciation in each zip-code
\end{itemize}


\subsection{Goodness-of-Fit Metrics}
In order to verify whether it's even sensible to try to calculate the impact of green energy projects, we also fit a null model, which omits turbine manipulation and solar manipulation.
This means that the null model's formula is:
\begin{verbatim}
change in home price v. last year ~ year + (1|zip_code)
\end{verbatim}
.

An ANOVA comparing these two models indicates that the full model is statistically significant (see figure \ref{anova}).
This indicates that it the turbine and solar manipulations are statistically significant.

\begin{figure}[h]
\centering
\includegraphics[width=0.9\linewidth]
{lmer_mod_anova.jpg} 
\caption{The full regression model is statistically significant, compared to the null model which omits green energy projects.}
\label{anova}
\end{figure}

The unadjusted intra-class correlation (ICC) is 0.011, which is quite low.
A high ICC would that a lot of the variance that is explained in the full model is done by the random-effects.
Rather, our fixed effects are explaining a lot of the variance-- and the effects we're trying to model are all fixed-effects.

\subsection{Estimates}

The fixed effects can be seen in figure \ref{fixef_general} and figure \ref{fixef_manipulation}, but they are listed below in their entirity.
Note that any variable whose 95\% CI includes 0.0 can plausibly be said to be negligible.

\begin{table}[]
\begin{tabular}{llll}
\textbf{Variable}                 & \textbf{Estimate} & \textbf{2.5\%} & \textbf{97.5\%} \\
(Intercept)                       & 0.053             & 0.047          & 0.058           \\
year2002                          & 0.015             & 0.013          & 0.016           \\
year2003                          & 0.014             & 0.013          & 0.015           \\
year2004                          & 0.021             & 0.02           & 0.023           \\
year2005                          & 0.047             & 0.046          & 0.049           \\
year2006                          & 0.05              & 0.049          & 0.052           \\
year2007                          & -0.013            & -0.014         & -0.012          \\
year2008                          & -0.071            & -0.072         & -0.069          \\
year2009                          & -0.131            & -0.132         & -0.13           \\
year2010                          & -0.116            & -0.117         & -0.115          \\
year2011                          & -0.09             & -0.091         & -0.089          \\
year2012                          & -0.097            & -0.098         & -0.096          \\
year2013                          & -0.029            & -0.031         & -0.028          \\
year2014                          & 0.005             & 0.004          & 0.006           \\
year2015                          & -0.012            & -0.013         & -0.01           \\
year2016                          & 0.001             & -0             & 0.002           \\
year2017                          & -0.014            & -0.016         & -0.013          \\
year2018                          & -0.003            & -0.004         & -0.002          \\
year2019                          & -0.003            & -0.004         & -0.002          \\
year2020                          & -0.006            & -0.008         & -0.005          \\
year2021                          & 0.055             & 0.054          & 0.057           \\
year2022                          & 0.082             & 0.081          & 0.083           \\
year2023                          & 0.015             & 0.014          & 0.016           \\
year2024                          & -0.03             & -0.031         & -0.029          \\
solar\_manipulation-2             & -0.001            & -0.004         & 0.003           \\
solar\_manipulation-3             & -0.001            & -0.005         & 0.002           \\
solar\_manipulation-4             & -0.002            & -0.005         & 0.002           \\
solar\_manipulation-5             & -0.003            & -0.007         & 0.001           \\
solar\_manipulation1              & 0.004             & 0              & 0.007           \\
solar\_manipulation2              & 0.004             & 0              & 0.007           \\
solar\_manipulation3              & 0.006             & 0.003          & 0.01            \\
solar\_manipulation4              & 0.01              & 0.006          & 0.014           \\
solar\_manipulation5              & 0.009             & 0.005          & 0.013           \\
solar\_manipulationCensored       & 0.004             & 0.001          & 0.006           \\
solar\_manipulationConstruction   & 0.003             & -0             & 0.006           \\
solar\_manipulationControl        & 0.006             & 0.001          & 0.011           \\
turbine\_manipulation-2           & 0                 & -0.006         & 0.007           \\
turbine\_manipulation-3           & 0                 & -0.006         & 0.007           \\
turbine\_manipulation-4           & 0.001             & -0.006         & 0.007           \\
turbine\_manipulation-5           & 0.001             & -0.006         & 0.008           \\
turbine\_manipulation1            & -0.001            & -0.007         & 0.004           \\
turbine\_manipulation2            & -0.004            & -0.01          & 0.002           \\
turbine\_manipulation3            & -0                & -0.006         & 0.006           \\
turbine\_manipulation4            & 0                 & -0.006         & 0.006           \\
turbine\_manipulation5            & -0.003            & -0.009         & 0.004           \\
turbine\_manipulationCensored     & 0.003             & -0.002         & 0.008           \\
turbine\_manipulationConstruction & 0.002             & -0.003         & 0.007          
\end{tabular}
\end{table}

Do note that while this model gives quite sensible estimates, we suffer from a relatively low sample size for a model of this complexity.
For this reason, this model was slightly rank-defficient.
(One notable missing value is turbine manipulation for the year just before operation.)

\begin{figure}[h]
\centering
\includegraphics[width=0.9\linewidth]
{fixef_general.png} 
\caption{Some of the fixed effects from the second study.}
\label{fixef_general}
\end{figure}

\begin{figure}[h]
\centering
\includegraphics[width=0.9\linewidth]
{fixef_manipulation.png} 
\caption{The rest of the fixed effects from the second study.}
\label{fixef_manipulation}
\end{figure}

Figure \ref{fixef_general} shows the annual intercepts for overall home appreciation in the national market.
Figure \ref{fixef_manipulation} shows what happens when solar and turbine projects are built in a zip-code.
I interpret these in the following way:
\begin{itemize}
\item test
\end{itemize}


\section{Appendix}

How did the rate of change of home appreciation change for zip-codes that had solar or turbine projects in them?

I think this study is really easy to understand with an example.
In 1979, Jimmy Carter put solar panels on his house, in zip-code 20500.
If we wanted to see if this solar panel project impacted the rate of home appreciation in zip-code 20500, we'd do the following:
Look at the year-over-year rate of home appreciation in 20500 in 1979, and for reference, include the previous few years (say, 1974-1979), and a few years after (say, 1979-1984).
Let's define the rate of home appreciation here as:

\noindent\textit{(current home price - last year's home price) / (last year's home price). }

Then we'd compare this to the median year-over-year rate of home appreciation of homes across the USA in zip-codes that didn't have any solar panels installed.
We'd see two trend lines-- that of our manipulation (zip-code 20500), and that of our control (all the other zip-codes without solar panels).

The problem with the above study is that our manipulation group has an n of 1, so it would be a good idea to expand our manipulation group to all other zip-codes in the USA that had solar panels installed in 1979.

This is the nature of study 1.
The results can be seen in figure \ref{study1solarfacets}.
\begin{figure}[h]
\centering
\includegraphics[width=0.9\linewidth]
{study1_solar_facets.png} 
\caption{Median change in home prices in zip-codes that had solar panels installed}
\label{study1solarfacets}
\end{figure}

Figure \ref{study1solarfacets} provides no useful insight, because it doesn't account for general changes in the real estate market.
For example, we can see the market tanking in 2008, but this completely confounds our ability to compare these lines to those a decade later.

Figure\ref{study1turbinefacets} shows us the DiD for turbine facilities, but it suffers from the same shortcomings as figure\ref{study1solarfacets}.
\begin{figure}[h]
\centering
\includegraphics[width=0.9\linewidth]
{study1_turbine_facets.png} 
\caption{Median change in home prices in zip-codes that had wind turbines installed}
\label{study1turbinefacets}
\end{figure}

The other issue with study 1 is small sample size (see figure \ref{study1samplesize}). 
The sample sizes for the manipulation groups pictured here are very small, only a fraction of the size of the control group.
In addition to this problem, while the observations in the manipulation group change in each of the diagrams, the observations in the manipulation group changes completely. 
This means that individual outliers in the control group might be skewing our results.
\begin{figure}[h]
\centering
\includegraphics[width=0.9\linewidth]
{study1_sample_size.png} 
\caption{The manipulation groups in study 1 were extremely small}
\label{study1samplesize}
\end{figure}



\end{document}



Here is the ANOVA comparing the two models. 
As you can see, ...



Here is the full summary as given by R:

And here are the fixed-effects, along with their 95% confidence-intervals.
We can interpret any fixed-effect whose 95% CI doesn't include zero as much more likely to be an observable phenomenon.

Interpretation:






















